\documentclass[12pt]{article}

% layout control
\usepackage[paper=a4paper,left=25mm,right=25mm,top=20mm,bottom=25mm]{geometry}
\usepackage[onehalfspacing]{setspace}
\setlength{\parskip}{.5em}
\usepackage{rotating}
\usepackage{setspace}
\usepackage{fancyhdr}
\usepackage{parallel}
\usepackage{parcolumns}

% math typesetting
\usepackage{array, amsmath, amsthm, amssymb, amsfonts, xfrac}

% tables
\usepackage{tabularx, booktabs, multicol, multirow, longtable}

% graphics stuff
\usepackage{subfig, graphicx, tikz}
\usepackage[space]{grffile} % allows us to specify directories that have spaces
\usepackage[section]{placeins} % prevents floats from moving past a \FloatBarrier or section
% \usepackage{pgfplots}

% Including External Code
\usepackage{verbatim, listings}
\lstset{
	language=R,
	basicstyle=\scriptsize\ttfamily,
	commentstyle=\ttfamily\color{gray},
	numbers=left,
	numberstyle=\ttfamily\color{gray}\footnotesize,
	stepnumber=1,
	numbersep=5pt,
	backgroundcolor=\color{white},
	showspaces=false,
	showstringspaces=false,
	showtabs=false,
	frame=single,
	tabsize=2,
	captionpos=b,
	breaklines=true,
	breakatwhitespace=false,
	title=\lstname,
	escapeinside={},
	keywordstyle={},
	morekeywords={}
	}

\usepackage{caption}
\DeclareCaptionFont{white}{\color{white}}
\DeclareCaptionFormat{listing}{\colorbox{gray}{\parbox{\textwidth}{#1#2#3}}}
\captionsetup[lstlisting]{format=listing,labelfont=white,textfont=white}

% -------------------- title -------------------- %

\title{}
\author{}
\date{}

\setlength{\headheight}{15pt}
\setlength{\headsep}{20pt}
\pagestyle{fancyplain}

\fancyhf{}

\lhead{\fancyplain{}{Anh Le}}
\chead{\fancyplain{}{Sta 601: Lab 10 - Hierarchical model}}
\rhead{\fancyplain{}{\today}}
\rfoot{\fancyplain{}{\thepage}}

%%%%%%%%%%%%%%%%%%%% DOCUMENT %%%%%%%%%%%%%%%%%%%%

% \doublespacing

\begin{document}

%\maketitle

\section*{Question b}

The result shows that the mean of weekly hours in all schools is 6.8 hours. The variance within school $\theta^{2} = 14.9$, which is higher than the variance between school $\tau^{2}=4.76$.

\begin{center}
\begin{tabular}{llll}
\toprule
& 2.5\% & 50\% & 97.5\% \\
$\sigma^{2}$ & 11.86874 & 14.90430 & 19.72093 \\
$\mu$ & 5.173789 & 6.784717 & 8.647925 \\
$\tau^{2}$ & 1.968522 & 4.761496 & 14.905368 \\
\bottomrule
\end{tabular}
\end{center}

The graphs below show  that we start from very diffuse prior and are able to get more concentrated posteriors.

\begin{center}
\includegraphics[scale=0.75]{posterior.png}
\captionof{figure}{Comparing the posterior and prior densities of $\sigma^{2}, \mu, \tau^{2}$}
\end{center}

\section*{Question c}
We see that the posterior variance between school is lower than the prior value. It is also small compared with the variance within school, the same result as above.

\begin{center}
\includegraphics[scale=0.7]{c}
\captionof{figure}{Comparing in-group and between-group variances}
\end{center}

\section*{Question d}

We have some evidence that the mean in school 7 is smaller than the mean in school 6:
\begin{equation}
Prob(\theta_{7} < \theta_{6}) = 0.5963
\end{equation}

There is no evidence that school 7 has the lowest mean:
\begin{equation}
Prob(\theta_{7}= min\theta_{j} ) = 1e-04
\end{equation}

\section*{Question e}

In Figure 2, we see that $\hat{\theta}$ is systematically smaller than $\hat{y}$. Coupled with the fact that there is more variance within group than between group, this suggests that our model may be erroneous.

\begin{center}
\includegraphics[scale=0.7]{e}
\captionof{figure}{Shrinkage of $\hat{\theta}$}
\end{center}

Concurring with the above result, $\hat{\mu}$ is quite small compared with $\bar{y_{i,j}}$:
\begin{equation}
\bar{y_{i,j}} = 7.766628 , \hat{\mu} = 6.814551
\end{equation}

\section*{My R Code}
\lstinputlisting{lab10.R}

\end{document} 
